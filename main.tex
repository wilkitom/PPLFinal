\documentclass{article}
\usepackage[utf8]{inputenc}
\usepackage{amsmath}
\usepackage{amssymb}
\usepackage{amsmath}

% adding a where clause - you can have multiple lines in where clause separated by "\\"
%\newcommand{\withWhere}[2]{$\begin{array}[t]{l}{{#1}}\\{}
%{{\;\;\;\;{\mathrm{where}}\;}}\\\;\;\;\;\;\;\begin{array}[t]{l}{#2}\end{array}\end{array}$}

% wwnlindent - for multiple where clauses - use wwnlindent for each extra line
%     see the definition of \classDRule below for an example usage.
\newcommand{\wwnlindent}{\newline{\mbox{\hspace{5.25em}}}}

% withWhere takes an enpression and then prints a properly 
% indented "where"  on the next line and then indents for the 
%"where" entry. See \intTSRule for an example of a usage with one 
% where clause. 
\newcommand{\withWhere}[1]{\begin{tabular}{l}{$#1$}\\{\hspace{3em}}where{\hspace{.5em}}\end{tabular}\wwnlindent}

% withWhere takes an enpression and then prints a properly 
% indented "where"  on the next line and then indents for the 
%"where" entry. See \intTSRule for an example of a usage with one 
% where clause. 
\newcommand{\withIf}[1]{\begin{tabular}{l}{$#1$}\\{\hspace{3em}}If{\hspace{.5em}}\end{tabular}\wwnlindent}

% a simple transition (with no \vdash)
\newcommand{\strans}[3]{\langle{#2}\rangle\,\rightarrow_{#1}\,{#3}}

% a transition (with  \vdash)
\newcommand{\trans}[4]{{#2}\,\vdash\,\langle{#3}\rangle\,\rightarrow_{#1}\,{#4}}

% ->(TS)  Type Structure transitions
%
\newcommand{\tsTrans}[3]{\trans{TS}{env}{{#1},\,{#3}}{({#2},\,{#3}')}}
\newcommand{\intTSTrans}{\tsTrans{Int}{Loc\; l}{sto}}
\newcommand{\intTSRule}{\noindent\withWhere{\intTSTrans}{(Loc\;l,sto')\,=\,alloc\;sto}}


% ->(D)
\newcommand{\declTrans}[2]{\strans{D}{{#1},\,env,sto}{#2}}
\newcommand{\varDTrans}[1]{\declTrans{var\, x\, :\, ts}{#1}}
\newcommand{\procDTrans}{\declTrans{procs\;p\;is\;s}{(env_0.proc[p\mapsto(s,f)],sto')}}

% argument to classDTrans is the expression to use as a return value
\newcommand{\classDTrans}[1]{\declTrans{class\;c\,=\,ts}{#1}}

% argument to classDTrans is the expression to use as a return value
\newcommand{\seqDTrans}[1]{\declTrans{D_1;\;D_2}{#1}}

\newcommand{\classDRule}{\noindent\
\withWhere{\classDTrans{env_o.classes[c\,\mapsto\,(ts,f)}}\
          {$f\;sto'\,=\,(tsval,sto'')$\wwnlindent$\tsTrans{ts}{tsval}{sto}$}}

% --------------- Stacked rules ------------------



\newcommand{\stackedRule}[2]{\begin{tabular}{c}{$#1$}\\\hline$#2$\end{tabular}}


\newcommand{\stackedRuleWhere}[3]{\begin{tabular}{l}{$#1$}\\\hline$#2$\\{\hspace{.125em}}where$\;{#3}$\end{tabular}\wwnlindent}

\newcommand{\stackedRuleIf}[3]{\begin{tabular}{l}{$#1$}\\\hline$#2$\\{\hspace{.125em}}If$\;{#3}$\end{tabular}\wwnlindent}

\newcommand{\varDRule}{\stackedRuleWhere
{\trans{TS}{env}{ts,sto}{(tsval,sto')}}
{\strans{D}{var\;x\;:\;ts,env,sto}{(env',sto')}}
{env' = \begin{array}[t]{l}case\;tsval\;of\\
                         \;\;(L\;l)  \; \rightarrow \; env_0.vars[x\mapsto{}l]\\
                         \;\;(E\; env'')\;\rightarrow\; env_0.records[x\mapsto{}env'']\\
                         \end{array}}}

\newcommand{\blockStm}{\stackedRule
{\;\begin{array}[t]{l}
   \strans{D}{D,env,sto}{(env',sto'')}\\
   \trans{Stm}{(env'\cup{}env)}{S,sto'}{sto'}
   \end{array}}
{\trans{Stm}{env}{begin\,D\; in\; S,sto}{sto'}}
}


{}






\title{PPLFinal}
\author{wilkitom }
\date{May 2019}

\begin{document}

\maketitle

\section{Introduction}
In this final, I will rewrite pages 134-142 of the Huttel text book. I will rewrite these pages using the Big-Step semantics of Dr.Caldwells code, changing the text to include one environment for all data structures and to include Type Structures. 

\section{Declarations}
\subsection{Formation Rules}
$X ::= x \mid r.X$\\
$P ::= p \mid r.P$\\
$a ::= n \mid X \mid a_1+a_2 \mid a_1*a_2 \mid a_1-a_2 \mid (a_1)$\\
$b ::= a_1 = a_2 \mid a_1 > a_2 \mid \neg b_1 \mid b_1 \wedge b_2$\\
$S ::= X := a \mid skip \mid S_1 ; S_2 \mid if \: b \: then \: S_1 \: else \: S_2 \mid while \: b \: do \: S \mid begin \: D \: S \: end \mid call \: X$\\
$TS ::= n \mid record \: D \mid X$\\
$D ::= var \: x := a;TS \mid proc \:p\: is\: S;TS \mid class\: c\: is \:D \:end;TS \mid D\: D$

\subsection{Environments}
$Env = (vars := Var \bigcup \{next\} \rightharpoonup Loc) \times (procs := Pnames \bigcup (S, Sto \vdash Sto') \times (records := Rnames\bigcup Env) \times (classes := Cnames \bigcup (TS \vdash TS', Sto \vdash Sto')$

\section{Big-Step Transition Rules}
\subsection{Generalized Variables}
The difference between the language in Huttle and the language we have been given to translate it too have few differences in this section. The only difference is that Env encompasses all 4 environments in Huttle. This being said, the only difference in the rules is accessing the elements of the single Env, instead of accessing all 4 types of environment separately. 
\\
$[GVAR-1_{BSS}]$
\stackedRuleWhere
{env'.records, env'.vars \vdash X \to l}
{env.records, env.vars \vdash r.X \to l}
{envr = env'}\\
\\
$[GVAR-2_{BSS}]$
\withWhere
{env.records,env.vars \vdash x \to l}
{\hspace{2cm}env.vars x = l}

\subsection{Arithmetic Expressions}
Similarly to the Generalized Variables rules, The difference between the language in Huttle and the language we have been given to translate it too have few differences in this section. The only difference is that Env encompasses all 4 environments in Huttle. This being said, the only difference in the rules is accessing the elements of the single Env, instead of accessing all 4 types of environment separately. \\
\\
$[GVAR_{BSS}]$
\stackedRuleWhere
{env'.records, env'.vars \vdash X \to l}
{env.records, env.vars, sto \vdash X \to_a v}
{sto \: l = v}\\
\\
$[PLUS_{BSS}]$
\stackedRuleWhere
{env'.records, env'.vars, sto \vdash a_1 \to_a v_1 \hspace{1cm}env'.records, env'.vars, sto \vdash a_2 \to_a v_2}
{env.records, env.vars, sto \vdash a_1+a_2 \to_a v}
{v= v_1 + v_2}\\
\\
$[MINUS_{BSS}]$
\stackedRuleWhere
{env'.records, env'.vars, sto \vdash a_1 \to_a v_1 \hspace{1cm}env'.records, env'.vars, sto \vdash a_2 \to_a v_2}
{env.records, env.vars, sto \vdash a_1-a_2 \to_a v}
{v= v_1 - v_2}\\
\\
$[MULT_{BSS}]$
\stackedRuleWhere
{env'.records, env'.vars, sto \vdash a_1 \to_a v_1 \hspace{1cm}env'.records, env'.vars, sto \vdash a_2 \to_a v_2}
{env.records, env.vars, sto \vdash a_1*a_2 \to_a v}
{v= v_1 * v_2}\\
\\
$[PARENT_{BSS}]$
\stackedRule
{env'.records, env'.vars, sto \vdash a_1 \to_a v_1}
{env'.records, env'.vars, sto \vdash (a_1) \to_a v_1}\\
\\
$[NUM_{BSS}]$
\withWhere
{env.records,env.vars,sto \vdash n \to_a v}
{\hspace{1cm}$\mathbb{N}$[\![n]\!]=v}
\subsection{Variable Declarations}



\end{document}
